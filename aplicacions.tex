\section{Aplicacions}
Com em comentat, els codis Reed-Solomon s'utilitzen en diferents àrees, com per
exemple els CDs, telèfons mòbils, sondes espacials, en comunicacions per
satèl·lit, en la transmissió digital de televisió, en sistemes xDSL de
comunicació per cable, etc.

En aquest apartat veurem diferents aplicacions del codis reed-solomon
i amb els paràmetres que s'utilitzen a cada cas. 

\subsection{CD}
Els CDs per la detecció i correció d'errors utilitza els Cross-Interleave
Reed-Solomon Code (CIRC). CIRC és un mètode de detecció i correció d'errors que
consisteix en afegir 8 bytes de paritat. Això es fa en dues etapes, anomenades
C1 i C2. En cada passa s'afegeixen 4 bytes de paritat. 

El calcul de la paritat es igual en cada cas. La única diferència es el nombre
de bytes utilitzats en cada cas per obtenir la paritat. A C1 tenim una
configuració de (32, 28), mentres que a C2 tenim (28, 24). Amb aquest esquema
podem corregir fins a 500 bytes. 

\subsection{DVD}
Els DVD utilitzen codis reed solomon, anomenats Reed-Solomon Product-Code
(RS-PC). Els RS-PC, al igual que els CIRC són la combinació de dos codis però
de major longuitud. En aquest cas tenim RS(208,192) i RS(182,172), on cada
simbol es 1 byte. Al ser més llargs, aquests codis ens proporcionaran un major
nombre de bits a poder corregir. Concretament, 2200 bytes. 
     
\subsection{Blu-Ray}
Els Blu-Ray utilitzen codis Reed-Solomon RS(248, 216). Cada paraula es de 248
bits, dels quals 216 son de dades i 32 de paritat.

\subsection{DVB}
Digital Video Broadcasting (DVB) és una organizació que promou estàndards
de televisión digital. 

Els sistemes DVB distribueixen les dades per satèlit, cable, etc. En totes les
variants s'utilitzen codis Reed Solomon per la detecció i control d'errors. La
seva configuració es la mateixa en totes les variants del DVB. 

En aquest cas, s'introdueixen 16 bytes de redundancia per cada paquet de 188
bytes de dades. Per tant, la configuració és RS(204, 188). 

 