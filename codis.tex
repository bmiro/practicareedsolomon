\section{Codis Reed-Solomon}
El Reed-Solomon és un codi cíclic no binari i constitueix una subclasse dels
codis BCH. Els codis cíclics son una subclasse dels codis de bloc estàndard de
detecció i correcció d'errors que protegeixen la informació contra errors en les
dades transmeses sobre un canal de comunicació. Aquest tipus de codi pertany a
la categoria FEC(\emph{Forward Error Correction}), és a dir, corregeix les dades
alterades al receptor i per fer-ho utilitza bits addicionals que permeten
aquesta recuperació a posteriori. 

El codi va ser inventat per Irving S. Reed y Gustave Solomon l'any 1960.
Actualment s'utilitza en àrees com els CDs, telèfons mòbils, sondes espacials,
en comunicacions per satèl·lit, en la transmissió digital de televisió, en
sistemes xDSL de comunicació per cable, etc. 

Un codificador Reed-Solomon agafa un bloc d'informació digital i afegeix bits
redundants. Els error poden sorgir durant la transmissió o emmagatzematge de
informació per diversos motius (renou, interferències, retxades en els discs,
etc.). El descodificador Reed-Solomon processa cada bloc i intenta corregir els
errors i recuperar la informació original. El nombre i tipus d'error que poden
ser corregits depenen de les característiques del codi Reed-Solomon. 

Un codi Reed-Solomon s'especifica com RS(n,k) amb símbols de \emph{s} bits. Això
significa que el codificador agafa \emph{k} símbols dels \emph{s} bits i afegeix
símbols de paritat per fer una paraula codi de \emph{n} símbols. Existeixen
$n-k$ símbols de paritat de \emph{s} bits cada un. Un descodificador pot corregir
fins a \emph{t} símbols que contenen error en una paraula codi, on $2t=n-k$. 

\subsection{Exemple}
Un codi popular Reed-Solomon és RS(255, 223) amb símbols de 8 bits.
Cada paraula codi conté 255 bytes de paraula codi, dels quals 223 bytes
son dades i 32 bytes son de paritat. 

Per aquest codi es té $N = 255, k = 223, s = 8$ i per tant: 

$$2t = 32, t = 16$$

El descodificador pot corregir qualsevol error de 16 símbols en la paraula codi,
és a dir, errors de fins a 16 bytes en qualsevol lloc de la paraula.