\section{Introducció}
En una transmissió d'informació per un canal existeix la possibilitat que la
informació enviada no arribi correctament al receptor. Durant la transmissió
poden produir-se errors per diferents motius i per això necessitam
mecanismes per saber si la informació rebuda es correspon amb l'enviada. 

El mecanisme més immediat per solucionar aquest problema és la simple
repetició de cada símbol transmès un nombre \emph{n} de vegades. Si els \emph{n}
símbols rebuts son iguals, es suposarà que en cap transmissió hi ha hagut error.
Si els \emph{n} símbols rebuts no son tots iguals, implicarà que hi ha hagut
error en alguna de les transmissions i s'identificarà com a símbol transmès el
que aparegui més vegades en la repetició. 
 
Quan més vegades s'envii el símbol la probabilitat d'error disminuirà, però a
costa d'incrementar el temps de transmissió. Aquest mecanisme planteja el
compromís entre la velocitat d'enviament dels missatges i la seva fiabilitat.
Per això, existeixen mètodes més sofisticats que aconsegueixen una major
velocitat de transmissió dels missatges mitjançant una codificació adequada. 

Aquest mètodes s'anomenen els codis detectors i correctors d'errors. La seva
funció és detectar i corregir erros en la transmissió d'un missatge sense
necessitat de repetir la transmissió. Bàsicament consisteix en incloure
informació redundant en la transmissió per permetre al receptor detectar
que s'ha produït un error i deduir quin és el caràcter que s'ha transmès. 

En aquest document, estudiarem un tipus de codi detector i corrector d'errors.
En concret, el codi que analitzarem  serà el Reed-Solomon. També veurem
exemples d'aplicació d'aquest codi sobre diferents tipus de sistemes
(dispositius d'emmagatzematge, telèfons mòbils, mòdems, etc). 